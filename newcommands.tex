\newcommand{\m}{\textrm}
\newcommand{\apx}{\approx}
\newcommand{\cd}{\cdot}
\newcommand{\del}{\partial}  % partielle Ableitung df
\newcommand{\pdiff}{\diffp} % partielle Ableitung df/dt
\newcommand{\tdiff}{\m{d}} % totales Differential df
\newcommand{\difft}{\m{d}} % totales Differential df
\newcommand{\gt}{\geqslant} % größer gleich
\newcommand{\lt}{\leqslant} % kleiner gleich
\newcommand{\PHI}{\varphi}
\newcommand{\EPS}{\varepsilon}
\newcommand{\sei}{\overset{!}{=}} % sei definiert als
\newcommand{\df}{\Rightarrow} % daraus folgt - Pfeil
\newcommand{\sgn}{\operatorname{sgn}}


\newcommand{\vs}[1]{\vspace{#1pt}}   %erzeugt mit \vs{<zahl>} die Eigabe \vspace{<zahl>pt}, also einen freien Raum mit der Höhe <zahl>pt
\newcommand{\hs}[1]{\hspace{#1pt}}

\newcommand{\anm}{\intertext} % Kommentare in align (linksbündige Textzeile)
\newcommand{\anr}[1]{\tag*{\bigg{|}{#1}}} % Kommentare rechtsbündig in align neben der geschriebenen Formel, ohne diese zu verrücken
\makeatletter
\newcommand{\subalign}[1]{% % anstelle von substack für aligned "&="
  \vcenter{%
    \Let@ \restore@math@cr \default@tag
    \baselineskip\fontdimen10 \scriptfont\tw@
    \advance\baselineskip\fontdimen12 \scriptfont\tw@
    \lineskip\thr@@\fontdimen8 \scriptfont\thr@@
    \lineskiplimit\lineskip
    \ialign{\hfil$\m@th\scriptstyle##$&$\m@th\scriptstyle{}##$\crcr
      #1\crcr
    }%
  }
}
\makeatother
%<<<<<<Elektronik>>>>>>
\newcommand{\recht}{
	\begin{tikzpicture}[scale=0.08]
		\draw (0,0)--(0,1);
		\draw (0,1)--(1,1);
		\draw (1,1)--(1,0);
		\draw (1,0)--(2,0);
		\draw (2,0)--(2,1);
	\end{tikzpicture}
} % Rechteckspannung

\DeclareRobustCommand{\gleich}{
	\begin{tikzpicture}[scale=0.08]
		\draw (0,0)--(2,0);
		\draw (0,1)--(0.4,1);
		\draw (0.8,1)--(1.2,1);
		\draw (1.6,1)--(2,1);
	\end{tikzpicture}
} % Gleichspannung

\newcommand{\sinus}{\backsim} % Sinusspannung
%\newcommand{\abs}[1]{\left|#1\right|}

\newcommand{\1}{\mathbbm{1}}

\definecolor{uniblau}{HTML}{004A99} 

% Matrizen
\newcommand{\mat}[4]{
	\begin{pmatrix}
		#1&#2\\
		#3&#4
	\end{pmatrix}
}

\newcommand{\vek}[2]{
	\begin{pmatrix}
		#1\\
		#2
	\end{pmatrix}
}

\def\CC{{C\nolinebreak[4]\hspace{-.05em}\raisebox{.4ex}{\tiny\bf ++}}} % C++ Symbol mit \CC

\newcommand{\dontmovesqrt}[1]{\smash[b]{#1}} % Den neuen Befehl \dontmovesqrt{} um den Befehl \underbrace{<eq>}_{<bla>} bei Wurzeln legen, um Wurzelform zu erhalten


\newcommand{\twocases}[3]{% 	% geschweifte Klammer um Inhalt der ersten beiden Klammern, rechts mit Anmerkung dahinter, zb. {x}{y}{=\text{const.}}
	\begin{rcases}%
		#1\\
		#2
	\end{rcases}%
	#3%
}
\newcommand{\threecases}[4]{%
	\begin{rcases}%
		#1\\
		#2\\
		#3
	\end{rcases}%
	#4%
}
\newcommand{\fourcases}[5]{%
	\begin{rcases}%
		#1\\
		#2\\
		#3\\
		#4
	\end{rcases}%
	#5%
}
\newcommand{\fivecases}[6]{%
	\begin{rcases}%
		#1\\
		#2\\
		#3\\
		#4\\
		#5
	\end{rcases}%
	#6%
}

% definition von \paragraph als "subsubsubsection"
\makeatletter
\renewcommand\paragraph{\@startsection{paragraph}{4}{\z@}%
            {-2.5ex\@plus -1ex \@minus -.25ex}%
            {1.25ex \@plus .25ex}%
            {\normalfont\normalsize\bfseries}}
\makeatother
\setcounter{secnumdepth}{4} % how many sectioning levels to assign numbers to
\setcounter{tocdepth}{4}    % how many sectioning levels to show in ToC

\newcommand{\name}{\text{Lennart Groß}}
\newcommand{\strasse}{\text{Gosener Str. 1}}
\newcommand{\stadt}{15732 Eichwalde}
\newcommand{\matrikel}{Matrikel-Nr.: 214204863}
\newcommand{\course}{Physik}
\newcommand{\betreuer}{Prof. Dr. rer. nat. Sylvia Speller\\Dr. rer. nat. Ingo Barke}
\newcommand{\Titelone}{Ionenleitfähigkeitsmikroskopie für nano-}
\newcommand{\Titeltwo}{morphologische Charakterisierung von Herzzellen}
\newcommand{\Type}{Bachelorarbeit}
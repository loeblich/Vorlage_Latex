\parindent0pt % setzt Einrückungen auf 0pt, verhindert sie also

%geometry
%\geometry{a4paper,left=20mm,right=15mm, top=15mm, bottom=15mm}

% paralist
\plitemsep=-10pt % Abstand zwischen den Listenpunkten

% \definecolor{sgreen}{HTML}{209d69}
% \definecolor{sred}{HTML}{914ccd}

\numberwithin{equation}{section}
\numberwithin{figure}{section}
\numberwithin{table}{section} % Zur Numerierung der Bilder entsprechend der Sections. Für subsections {table}{subsection} schreiben

\newcolumntype{L}{>{$}l<{$}} % Definiert in Tabellen L als Spalte, in der automatisch der math mode angewendet wird // schreibt man x^2, wird daraus $x^2$
\newcolumntype{R}{>{$}r<{$}} % s.o. für r
\newcolumntype{C}{>{$}c<{$}} % s.o. für c

\newcolumntype{F}{>{$\collectcell\bf}c<{\endcollectcell$}} % für zentrierte Spalten, die automatisch Fett ausgegeben werden

% \clubpenalty = 1000 
% \widowpenalty = 1000 
% \displaywidowpenalty = 1000 

%\bibliographystyle{abbrvnat}
\bibliographystyle{unsrtnat}